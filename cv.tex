% Options for packages loaded elsewhere
% Options for packages loaded elsewhere
\PassOptionsToPackage{unicode}{hyperref}
\PassOptionsToPackage{hyphens}{url}
\PassOptionsToPackage{dvipsnames,svgnames,x11names}{xcolor}
%
\documentclass[
  letterpaper,
  DIV=11,
  numbers=noendperiod]{scrartcl}
\usepackage{xcolor}
\usepackage[margin=1in]{geometry}
\usepackage{amsmath,amssymb}
\setcounter{secnumdepth}{-\maxdimen} % remove section numbering
\usepackage{iftex}
\ifPDFTeX
  \usepackage[T1]{fontenc}
  \usepackage[utf8]{inputenc}
  \usepackage{textcomp} % provide euro and other symbols
\else % if luatex or xetex
  \usepackage{unicode-math} % this also loads fontspec
  \defaultfontfeatures{Scale=MatchLowercase}
  \defaultfontfeatures[\rmfamily]{Ligatures=TeX,Scale=1}
\fi
\usepackage{lmodern}
\ifPDFTeX\else
  % xetex/luatex font selection
\fi
% Use upquote if available, for straight quotes in verbatim environments
\IfFileExists{upquote.sty}{\usepackage{upquote}}{}
\IfFileExists{microtype.sty}{% use microtype if available
  \usepackage[]{microtype}
  \UseMicrotypeSet[protrusion]{basicmath} % disable protrusion for tt fonts
}{}
\makeatletter
\@ifundefined{KOMAClassName}{% if non-KOMA class
  \IfFileExists{parskip.sty}{%
    \usepackage{parskip}
  }{% else
    \setlength{\parindent}{0pt}
    \setlength{\parskip}{6pt plus 2pt minus 1pt}}
}{% if KOMA class
  \KOMAoptions{parskip=half}}
\makeatother
% Make \paragraph and \subparagraph free-standing
\makeatletter
\ifx\paragraph\undefined\else
  \let\oldparagraph\paragraph
  \renewcommand{\paragraph}{
    \@ifstar
      \xxxParagraphStar
      \xxxParagraphNoStar
  }
  \newcommand{\xxxParagraphStar}[1]{\oldparagraph*{#1}\mbox{}}
  \newcommand{\xxxParagraphNoStar}[1]{\oldparagraph{#1}\mbox{}}
\fi
\ifx\subparagraph\undefined\else
  \let\oldsubparagraph\subparagraph
  \renewcommand{\subparagraph}{
    \@ifstar
      \xxxSubParagraphStar
      \xxxSubParagraphNoStar
  }
  \newcommand{\xxxSubParagraphStar}[1]{\oldsubparagraph*{#1}\mbox{}}
  \newcommand{\xxxSubParagraphNoStar}[1]{\oldsubparagraph{#1}\mbox{}}
\fi
\makeatother


\usepackage{longtable,booktabs,array}
\usepackage{calc} % for calculating minipage widths
% Correct order of tables after \paragraph or \subparagraph
\usepackage{etoolbox}
\makeatletter
\patchcmd\longtable{\par}{\if@noskipsec\mbox{}\fi\par}{}{}
\makeatother
% Allow footnotes in longtable head/foot
\IfFileExists{footnotehyper.sty}{\usepackage{footnotehyper}}{\usepackage{footnote}}
\makesavenoteenv{longtable}
\usepackage{graphicx}
\makeatletter
\newsavebox\pandoc@box
\newcommand*\pandocbounded[1]{% scales image to fit in text height/width
  \sbox\pandoc@box{#1}%
  \Gscale@div\@tempa{\textheight}{\dimexpr\ht\pandoc@box+\dp\pandoc@box\relax}%
  \Gscale@div\@tempb{\linewidth}{\wd\pandoc@box}%
  \ifdim\@tempb\p@<\@tempa\p@\let\@tempa\@tempb\fi% select the smaller of both
  \ifdim\@tempa\p@<\p@\scalebox{\@tempa}{\usebox\pandoc@box}%
  \else\usebox{\pandoc@box}%
  \fi%
}
% Set default figure placement to htbp
\def\fps@figure{htbp}
\makeatother





\setlength{\emergencystretch}{3em} % prevent overfull lines

\providecommand{\tightlist}{%
  \setlength{\itemsep}{0pt}\setlength{\parskip}{0pt}}



 


% Sans-serif body, no page numbers, compact lists, tidy sections
\renewcommand{\familydefault}{\sfdefault}
\pagestyle{empty}
\setlength{\parindent}{0pt}
\setlength{\parskip}{5pt}
\usepackage{enumitem}
\setlist{nosep, leftmargin=*}
\RedeclareSectionCommand[
  beforeskip=8pt plus 3pt minus 3pt,
  afterskip=4pt plus 2pt minus 2pt,
  font=\large\bfseries
]{section}
\RedeclareSectionCommand[
  beforeskip=6pt plus 2pt minus 2pt,
  afterskip=3pt plus 1pt minus 1pt,
  font=\normalsize\bfseries
]{subsection}
\KOMAoption{captions}{tableheading}
\makeatletter
\@ifpackageloaded{caption}{}{\usepackage{caption}}
\AtBeginDocument{%
\ifdefined\contentsname
  \renewcommand*\contentsname{Table of contents}
\else
  \newcommand\contentsname{Table of contents}
\fi
\ifdefined\listfigurename
  \renewcommand*\listfigurename{List of Figures}
\else
  \newcommand\listfigurename{List of Figures}
\fi
\ifdefined\listtablename
  \renewcommand*\listtablename{List of Tables}
\else
  \newcommand\listtablename{List of Tables}
\fi
\ifdefined\figurename
  \renewcommand*\figurename{Figure}
\else
  \newcommand\figurename{Figure}
\fi
\ifdefined\tablename
  \renewcommand*\tablename{Table}
\else
  \newcommand\tablename{Table}
\fi
}
\@ifpackageloaded{float}{}{\usepackage{float}}
\floatstyle{ruled}
\@ifundefined{c@chapter}{\newfloat{codelisting}{h}{lop}}{\newfloat{codelisting}{h}{lop}[chapter]}
\floatname{codelisting}{Listing}
\newcommand*\listoflistings{\listof{codelisting}{List of Listings}}
\makeatother
\makeatletter
\makeatother
\makeatletter
\@ifpackageloaded{caption}{}{\usepackage{caption}}
\@ifpackageloaded{subcaption}{}{\usepackage{subcaption}}
\makeatother
\usepackage{bookmark}
\IfFileExists{xurl.sty}{\usepackage{xurl}}{} % add URL line breaks if available
\urlstyle{same}
\hypersetup{
  pdftitle={Curriculum Vitae},
  colorlinks=true,
  linkcolor={blue},
  filecolor={Maroon},
  citecolor={Blue},
  urlcolor={Blue},
  pdfcreator={LaTeX via pandoc}}


\title{Curriculum Vitae}
\author{}
\date{}
\begin{document}
\maketitle


\section{Jiale Zhao}\label{jiale-zhao}

\begin{itemize}
\tightlist
\item
  Location: Beijing, China
\item
  Phone: +86 182-5673-1893
\item
  Email: jialeuuz@gmail.com
\item
  GitHub: github.com/jialeuuz
\end{itemize}

\subsection{Professional Summary}\label{professional-summary}

I have two years of LLM algorithm internship experience, with strong
coding and research skills in LLM and multimodal domains. Initiated ML
research in a university lab (Freshman), conducted CV research
(Sophomore Fall), transitioned to NLP research (Junior Spring), and
gained industry experience as an LLM Algorithm Intern at Li Auto (Junior
Year).

Key interests: 1) Agent‑based LLMs; 2) Data \& self‑improving systems;
3) Human--LLM interaction; 4) Interpretability \& analysis; 5)
Benchmarks \& evaluation; 6) Bridging research and real applications.

\subsection{Experience}\label{experience}

\subsubsection{LLM Algorithm Intern --- Li Auto (Sep 2023 -- present ·
Beijing)}\label{llm-algorithm-intern-li-auto-sep-2023-present-beijing}

\begin{itemize}
\tightlist
\item
  Data Flywheel for Code LLM: Iterative cycle centered on evaluation
  (SFT → evaluation → data generation → filtering → back to SFT) to
  mass‑produce high‑quality training and evaluation data.
\item
  Multi‑step Reasoning + Tool Use Agent: Constructed SFT data for LLM
  Q\&A, implemented API function calls, and solved complex reasoning
  problems through multi‑step processes.
\item
  MindGPTo: End‑to‑end multimodal app inspired by GPT‑4o with
  paralinguistic features; built from scratch with modular FE/BE,
  large‑scale audio data pipelines, and SFT to enhance conversational
  capabilities.
\end{itemize}

\subsection{Education}\label{education}

\subsubsection{Chongqing Univ. of Posts and Telecommunications ---
B.Eng., Algorithm Engineering
(2021--2025)}\label{chongqing-univ.-of-posts-and-telecommunications-b.eng.-algorithm-engineering-20212025}

Initiated ML research in a university lab (Freshman), conducted CV
research (Sophomore Fall), transitioned to NLP research (Junior Spring),
and gained industry experience as an LLM Algorithm Intern at Li Auto
(Junior Year).

\subsection{Publications (Under
Review)}\label{publications-under-review}

\begin{itemize}
\tightlist
\item
  ThinkPilot (under review): Submitted to AACL 2025 via ARR (plan to
  resubmit to ACL 2025 after revision based on reviews). Contributions:
  experiment design and implementation; explainability integration;
  appendix and parts of main text.
\item
  Decoding the Ear (under review): arXiv:2510.20513.
\item
  ExpressiveSpeech Dataset: \textasciitilde51h bilingual expressive
  speech (\textasciitilde14k utt.).
\end{itemize}

\subsection{Research Directions}\label{research-directions}

\subsubsection{Agent‑based LLMs --- Efficient Complex Problem
Solving}\label{agentbased-llms-efficient-complex-problem-solving}

Many real‑world problems require multi‑source reasoning and dynamic user
feedback. I focus on: 1) decomposing complex tasks; 2) time‑frame
mechanisms for updates and reflection; 3) coordinating several parallel
smaller models (e.g., multiple 7B LLMs) to improve efficiency/accuracy
at the same compute.

\subsubsection{Self‑Evaluation →
Self‑Improvement}\label{selfevaluation-selfimprovement}

Three failure modes in code tasks: (1) inconsistently solved problems;
(2) problems rarely solved spontaneously but solvable with simple hints;
(3) problems needing complex guidance. Address via sampling,
heuristics/self‑guided strategies, and iterative evolution. Continuous
improvement needs diverse fresh data and strong filtering; diverse,
accurate test queries are critical.

\subsubsection{Human--LLM Interaction --- Enabling LLMs to
Ask}\label{humanllm-interaction-enabling-llms-to-ask}

Enable models to challenge incorrect or contradictory information,
detect and request missing conditions for underspecified problems, and
proactively seek user assistance on hard tasks.

\subsubsection{Interpretability \& Analysis --- Insight → Control →
Safety}\label{interpretability-analysis-insight-control-safety}

Use deeper interpretability to develop more effective control over
reasoning. Explore lightweight control modules trained on large response
corpora to steer reasoning without modifying LLM weights.

\subsubsection{Benchmarks \& Evaluation --- Making Benchmarks
``Dumber''}\label{benchmarks-evaluation-making-benchmarks-dumber}

Transform single‑turn benchmarks into simulated multi‑turn interactions
featuring omissions, errors, and colloquial phrasing; for knowledge
tasks, intentionally vague queries requiring elicitation. Better matches
real user inputs.

\subsubsection{Bridging Research \&
Application}\label{bridging-research-application}

Keep academic research grounded in real‑world needs; pursue a PhD and
future faculty path while collaborating with industry to overcome
practical constraints like compute.




\end{document}
